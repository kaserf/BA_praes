\documentclass[compress]{beamer}

\usepackage[utf8]{inputenc}
\usepackage{graphicx}
\usepackage{color}
\usepackage{colortbl}
\usepackage{listings}
\usepackage{caption}
\usepackage{pgfpages}
\usepackage{tabularx}
\usepackage{amsthm}
\usepackage{url}

%\usetheme[basergb={1,0.2,0.2}]{Dcon}
%\usetheme[baseRGB={23,102,34}]{Dcon}
\usetheme{Dcon} % default color

%%%%%%%%%%%%%%%%%%%%%%%%%%%%% Definitions %%%%%%%%%%%%%%%%%%%%%%%%%%%%%%%%%%%%%

\newcommand{\logotum}{\includegraphics[scale=0.9]{TUMINF.pdf}}
\newcommand{\logoinf}{\includegraphics[scale=0.5]{IN_schwarz_CMYK.pdf}}

\newcommand{\hl}[1]{\fcolorbox{red}{white}{#1}}
\newcommand{\todo}[1]{\hl{TODO: #1...}}
\newcommand{\review}[0]{\todo{REVIEW}}
\newcommand{\comm}[1]{\textsuperscript{\bf \color{red}{\tiny [#1]}}}
\newcommand{\q}[0]{\comm{?}}
\newcommand{\s}[0]{\comm{*}}

\newcommand{\scite}[1]{\textsuperscript{\tiny\cite{#1}}}

\bibliographystyle{apalike}


\titlegraphic{\logotum}
\title[Intelligent Learning Framework]
{Intelligent Support\\ \strut for non-linear Serious Games}
\subtitle{\scriptsize{Bachelor's Thesis in Computer Science}}
\author[Felix Kaser]{Felix Kaser\\\tiny{\texttt{kaserf@in.tum.de}}}
\institute[TUM]
{Chair for Applied Software Engineering\\
Faculty of Informatics\\
Technische Universit\"at M\"unchen}
\date{July 1, 2010}
\logo{\logoinf}
\additionaltext{\tiny
{Supervisor: Prof. Bernd Brügge, PhD.\\
\hspace{7mm}Advisors: Dipl.-Inf. Univ. Dennis Pagano,\\
\hspace{17.2mm}Dipl.-Inf. Univ. Damir Ismailović}}

%%%%%%%%%%%%%%%%%%%%%%%%%%%%% Document %%%%%%%%%%%%%%%%%%%%%%%%%%%%%%%%%%%%%%%%%

\begin{document}

% title page
\begin{frame}[plain]
\titlepage
\end{frame}

\begin{frame}{Outline}
\tableofcontents
\end{frame}

\section{Introduction}

\subsection{Problem Statement}

\begin{frame}{Problem Statement}
games, learning, politics, etc.
\end{frame}

\subsection{Serious Games}

\begin{frame}{Overview}
what are serious games?
serious games are a bussines quote
\end{frame}

\subsection{Non Linear Games}

\begin{frame}{Frame 3}
What do I mean with non-linear games?\\
What are the problems when supporting non-linear games?\\
\end{frame}

\begin{frame}{Examples}
example of a serious game
\begin{block}{Physics}
block one
\end{block}
\begin{block}{Food Force}
block two
\end{block}
\end{frame}

\section{Intelligent Learning Framework}

\subsection{Overview}

\begin{frame}{Goals}
show whats the goal of the framework
\end{frame}

\begin{frame}{Pedagogical Agents}
short introduction of pedagogical agents
\end{frame}

\begin{frame}{Game - Framework Communication}
how it connects with the game
\end{frame}

\subsection{From Actions To Problems}

\begin{frame}{Use Cases}
use cases
\end{frame}

\begin{frame}{Topology}
topology action
\end{frame}

\begin{frame}{Action, Interaction, Event}
interaction, event
\end{frame}

\begin{frame}{Flow of Events}
flow of events through the system
\end{frame}

\begin{frame}{Problem Detection}
pattern matching with state machine, regex, blurry match ---> object design
\end{frame}

\section{Outlook}

\subsection{Future Work}

\begin{frame}{Future Work}
language to describe interactions and patterns\\
use AI / machine learning to learn interactions and patterns\\
extend the framework with above technologies\\
test the framework with various games
\end{frame}

\subsection{OLPC}

\begin{frame}{One Laptop Per Child}
explain the project, whats the connection to ILF
\end{frame}

\begin{frame}{Developers Programme}
how does it work, what do they expect, what do they provide
\end{frame}



\appendix
\pagenumbering{Roman}

%%%%%%%%%%%%%%%%%%%%%%%%%%%%%%%%% Appendix %%%%%%%%%%%%%%%%%%%%%%%%%%%%%%%%%%%%%%%%%%

\section{Bibliography}
\begin{frame}[allowframebreaks]{Literature}
\tiny{\bibliography{bibliography}}
\end{frame}

\section{Additional Material}
\begin{frame}

\end{frame}

\end{document}
